\documentclass[12pt]{article} % тип документа
\usepackage[T2A]{fontenc} % Поддержка русских букв
\usepackage[utf8]{inputenc} % кодировка 
\usepackage[english,russian]{babel} % языковой пакет
\usepackage{amsmath,amsfonts} % шрифты
\usepackage[colorlinks,urlcolor=blue]{hyperref} % для гиперссылок
\usepackage{anysize} % для отступов	
\usepackage{setspace} % интервал
\usepackage{dsfont} % индикатор
\onehalfspacing % уточняет интервал
\usepackage{indentfirst} % чтобы первый абзац в разделе тоже был с отступом
\marginsize{2.5cm}{2cm}{1cm}{1cm} % поля

\title{Пособие по скейтбордингу}
\author{Richie Jackson}

\begin{document}

\maketitle % создаёт заголовок с автором, названием и датой компиляции, можно убрать эту команду и самому сделать заголовок. Скорее всего эта команда будет неактульна для тебя, т.к. тебе будет нужен титульник. Титульник обычно делается в отдельном файлике и на него ссылаются в коде основного.

\section{Введение.Базовые формулы} % раздел

\par % команда для абзаца

Рассчитаем радость Олега от выполнения трюка на примере бонлесса: H --- happiness, B --- количество бонлессов.Формула: $H = -\alpha B_{oneless}^{2} + \beta B_{oneless}$. Для наглядности выпишем формулу на отдельной строке:

% Дефис между словами из-за, интервал числе 3--7, тире в предложении ---.

\[
H = -\alpha B_{oneless}^{2} + \beta B_{oneless}
\]

Отрицательная квадратичная зависимость объясняется тем, что помимо бонлессов, Олег, желая видеть разнообразие в катании, хочет посмотреть на выполнение \textit{файкрекеров} и \textit{хиппи джампов}. Заметим, что аналогичная зависимость между радостью и количеством трюков верна для всех трюков. 

\par

Наибольшую радость от выполнения бонлесса Олег получает в точке:
\[
Boneless_{max} = \frac{\beta}{2\alpha}
\]

\section{Трюки}

В этом разделе мы хотели бы привести наиболее полный список самых зрелищных трюков. 

\begin{enumerate} % список с номерами, itemize --- с точечками

\item Hippie jump
\item Boneless
\item Firecracker
\item Caveman
\item Handflip
\item Ollie north
	
\end{enumerate}

\section{Споты}

Раздел посвящен лучшим спотам для катания

\begin{itemize}
\item клумба
\item столб для знака
\item Рязань
\end{itemize}

\section{Полезные формулы}

Методы оценки крутости серии трюков

\begin{equation}
\begin{split} %позволяет разбить формулу на строки
Coolness_{series} = \sum\limits_{i\in series} \mathds{1}\{i = hippiejump\} + \mathds{1}\{i = boneless\} + \\
 + \mathds{1}\{i = firecracker\} + \mathds{1}\{i = handflip\} + \mathds{1}\{i = ollienorth\}  \}
\end{split}
\end{equation}

\end{document}