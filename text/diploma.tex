\documentclass[12pt]{article} % тип документа
\usepackage[T2A]{fontenc} % Поддержка русских букв
\usepackage[utf8]{inputenc} % кодировка 
\usepackage[english,russian]{babel} % языковой пакет
\usepackage{amsmath,amsfonts} % шрифты
\usepackage[colorlinks,urlcolor=blue]{hyperref} % для гиперссылок
\usepackage{anysize} % для отступов	
\usepackage{setspace} % интервал
\usepackage{dsfont} % индикатор
\onehalfspacing % уточняет интервал
\usepackage{indentfirst} % чтобы первый абзац в разделе тоже был с отступом
\marginsize{2.5cm}{2cm}{1cm}{1cm} % поля

\begin{document}
\begin{abstract}
hello
\end{abstract}

\section{Введение}
\par Рекомендательные системы представляют из себя программы, которые помогают пользователям в выборе подходящим им объектов из некоторой библиотеки.  %TODO: найти статью где будет введено это понятие и сосласться на нее, можно использовать [1] из iterative smooting technology %
Это могут быть фильмы(Netflix), товары(Amazon), музыкальные исполнители(Last.fm). В процессе вычисления рекомендаций эти системы используют оценки, сделанные пользователем. 


\section{Обзор литературы}

%чтото из эбстректа, введения и заключения а не дебри и считальные вещи

Тут говорю про статьи:
кластеризация

А) русская статья где сначала хеш-кластеризация потом ближашие соседи
Б) кластеризация с пчелиным алгоритом

итеративность

В) итеративное сглаживание ( где по одному выкидывается)
Г) итеративное коллаборейтив фильтеринг где много раз перерасчитываются $r_{ui}$ 

В моей работе был построен алгоритм, сочетающий в себе две предыдущие категории подходов: прпесчет рейтингов происходит в зависимости от "качества" проведенной
на базе этих рейтингов кластеризации оюъектов


\section{Методология и формулировка задачи}
\subsection{Формулировка задачи}
\par
Имеется конечный набор объектов $I=\{i_{1}, i_{2}, \dots, i_{m}\}$ и конечный набор пользователей $U=\{u_{1}, u_{2}, \dots, u_{n}\}$
Положим $r_{ui}$ - рейтинг, который пользователь $u\in U$ поставил объекту $i\in I$. Тогда n x m матрицу  $R = (r_ui) $ будем называть мартицей рейтинговю. Задачей рекомендательной системы является оценка неизвестных рейтингов $r_{ui}$. Главной особенностью рекомендательных систем является высокая разреженность мартицы R.
Далее пользователю рекомендуются рейтинги с самой высокой оценкой. 

\subsection{Метрики сходства}
Важным шагов при построении рекомендательной системы является задание функции похожести между пользователями и/или объектами. Ниже приведены наиболее часто встречающиеся ффункции похожести пользователей, функции похожести объектов вводятся аналогично.
\begin{itemize}

\item{Корреляция Пирсона}
%TODO: тут вставить почему она крутая(как просил Рыжов))
\[
s_{uv} = \frac{\sum_{i \in I_{uv}} (r_{ui} - \bar{r}_{ui})}{\sqrt{\sum_{i \in I_{uv}}  (r_{ui} - \bar{r}_{u})^2} \sqrt{\sum_{i \in I_{uv}}  (r_{vi} - \bar{r}_{v})^2}}
\]
\item{Косинусная метрика}

\[
s_{uv} = \frac{\sum_{i \in I_{uv}} r_{ui}r_{vi}} {\sqrt{\sum_{i \in I_{uv}} r_{ui}^2} \sqrt{\sum_{i \in I_{uv}} r_{vi}^2}}
\]

\vspace*{2\baselineskip}
\item{Другие метрики}
\[s_{uv} = \frac{|\{i \in I: r_{ui} = r_{vi}\}|}{\sqrt{\sum_{i \in I_{uv}} (r_{ui} - r_{vi})^2 }}
\]
\end{itemize}
\subsection{Нечеткая кластеризация}



\section{Метод Итерационного улучения качества классификации}
\subsection{Описание алгоритма}
Качество существующих рекомендательных систем, используюущих кластеризацию пользователей, можетсильно страдать из-за высокой разреженности выходных данных. Некоторые алгоритмы борются с этой проблемой путем заполнения пропущенных значений стандартными, например, нулями или средним значением рейтинга, проставленным пользователем или среднего значения, проставленного объекту. Ниже приведен алгоритм, решающий проблему разреженности.

\begin{enumerate} 

\item Выбирается метрика, используемая для рассчета расстояния между пользователями. Далее вычисляется матрица расстояний между пользователями


\item Кластеризация пользователей
С помощью алгоритма нечетких к-средних множество пользователей разбивается на классы $\{k_{1},\dots,k_{p}\}=K$
Каждому пользователю $u \in U$ приписывается вектор $\lambda(u)=(\lambda_{1}(u),\dots,\lambda_{p}(u)), \lambda(u)\in\mathbb{N}, p=|K|$

Введем понятие качества кластеризации пользователя $u \in U$ :
$\varphi_{p}(u) = \min_{j} \{k_{i_{j}} \in K: \sum_{j} \lambda_{i_{j}}(u) >= p\}, 0 < p < 1$ 

\item 

\end{enumerate}

\subsection{Дополнительные параметры}

\begin{itemize}

\item Начальное количество кластеров K
в какойто-статье сказано, что ок брать $K <= \frac{\sqrt{n}}{2} $
\item Пороговое качество кластеризации p
\item Сдвиг числа кластеров $\delta K$
\item Пороговое число совместно отрейтингованых объектов M

\end{itemize}

\section{Тестирование}
\subsection{Способы оценки рекомендательных систем}
\begin{itemize}
\item  Качество 
\par Стандартными оценками качества работы рекомендательного алгоритма являются среднеквадратичное отклоенение (Root mean squared error, RMSE) и среднее абсолютное отклоение (Mean absolute error, MAE)
\[
	RMSE=\sqrt{\frac{1}{|T|}\sum_{(u,i)\in T} (\hat{r}_{ui} - r_{ui})^2}
\]
\[	
	MAE=\sqrt{\frac{1}{|T|}\sum_{(u,i)\in T} |\hat{r}_{ui} - r_{ui}|}
\]
\item Стабильность
\par Стабильность рекомендательной системы показывает, насколько оценки, полученные системой для конкретных объектв, изменяются при появлении новых пользовательских оценок, полностью или частично совпадающих с предыдущими оценками, данными системой. Таким образом, метод оценки стабильности системы состоит из двух шагов: на первом %TODO:найти ссылку на статью где вводился метод - взять [2] bpb из статьи про стабильность %
шаге система рассчитывает оценки неизвестных рейтингов на некоторых входных данных. Затем случайное подмножество полученных рекомендаций добавляется к начальным входным данным. На втором шаге алгоритм рассчитывает оценки на расширенных входных данных, после чего сравниваются оценки, полученные алгоритом для оставшихся рейтингов на обоих шагах. Мера называется Среднеквадратичный сдвиг (Root mean squared shift, RMSS)
\[
	RMSS=\sqrt{\sum_{(u,i) \in P_{1} \cap P_{2})} (P_{1}(u,i) - P_{2}(u,i))^2 /  |P_{1} \cap P_{2}|}
\]  
где $P_{1}$ и $P_{2}$ - рекомендации, полученные на шагах 1 и 2 соответственно.
\end{itemize}
\subsection{Проверочные данные}
Для проверки качества работы алгорита были использованы данные сайта http://movielens.com. В выборке содержались данные о 100000 оценкок, проставленных  943 пользователями 
1664 фильмам. Каждый пользователь оценил как минимум 20 фильмов. Этот набор данных вместе с набором данных, предоставленным компанией Netflix является стандартным для оценки качества работы рекомендательных систем и используется в большинстве работ на эту тему.
В исходном наборе разреженность матрицы рейтингов составляла 6,3\% , при тестировании она искуственно понимажалась до 1\%


\subsection{Процесс тестирования}
В процессе тестирования алгоритм сравнивался со следующими:
\begin{itemize}
\item SVD
\item Рекомендация наиболее популярных объектов
\item Коллаборативная фильтрация, основанная на пользователях
\item Коллаборативная фильтрация, основанная на объектах
содержимое...
\end{itemize}


\subsection{Результаты}


\section{Литература}



\end{document}

