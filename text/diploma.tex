\documentclass[12pt]{article} % тип документа
\usepackage[T2A]{fontenc} % Поддержка русских букв
\usepackage[utf8]{inputenc} % кодировка 
\usepackage[english,russian]{babel} % языковой пакет
\usepackage{amsmath,amsfonts} % шрифты
\usepackage[colorlinks,urlcolor=blue]{hyperref} % для гиперссылок
\usepackage{anysize} % для отступов	
\usepackage{setspace} % интервал
\usepackage{dsfont} % индикатор
\onehalfspacing % уточняет интервал
\usepackage{indentfirst} % чтобы первый абзац в разделе тоже был с отступом
\marginsize{2.5cm}{2cm}{1cm}{1cm} % поля

\begin{document}
\begin{abstract}
hello
\end{abstract}

\section{Введение}
\par

\section{Обзор литературы}

%чтото из эбстректа, введения и заключения а не дебри и считальные вещи

Тут говорю про статьи:
кластеризация

А) русская статья где сначала хеш-кластеризация потом ближашие соседи
Б) кластеризация с пчелиным алгоритом

итеративность

В) итеративное сглаживание ( где по одному выкидывается)
Г) итеративное коллаборейтив фильтеринг где много раз перерасчитываются r_{ui} 

В моей работе был построен алгоритм, сочетающий в себе две преыдущие категории подходов: прпесчет рейтингов происходит в зависимости от "качества" проведенной
на базе этих рейтингов кластеризации оюъектов


\section{Методология и формулировка задачи}
\par
Имеется конечный набор объектов $I=\{i_{1}, i_{2}, ..., i_{n}\}$ и конечный набор пользователей $U=\{u_{1}, u_{2}, ..., u_{n}\}$

\par
Обозначим $I_{u}= $

\subsection{Метрики сходства}

$s_{uv} = \frac{\sum_{i \in I_{uv}} (r_{ui} - \bar{r}_{ui})}{\sqrt{\sum_{i \in I_{uv}}  (r_{ui} - \bar{r}_{u})^2} \sqrt{\sum_{i \in I_{uv}}  (r_{vi} - \bar{r}_{v})^2}}$

\vspace*{2\baselineskip} %  вставить пустые строки

$s_{uv} = \frac{\sum_{i \in I_{uv}} r_{ui}r_{vi}} {\sqrt{\sum_{i \in I_{uv}} r_{ui}^2} \sqrt{\sum_{i \in I_{uv}} r_{vi}^2}}$

\vspace*{2\baselineskip}

$s_{uv} = \frac{|\{i \in I: r_{ui} = r_{vi}\}|}{\sqrt{\sum_{i \in I_{uv}} (r_{ui} - r_{vi})^2 }}$




\section{Итеративное улучшение качества кластеризации}

\section{Тестирование}
\subsection{Способы оценки рекомендательных систем}

$RMSE=\sqrt{\frac{1}{|T|}\sum_{(u,i)\in T} (\hat{r}_{ui} - r_{ui})^2}$

$MAE=\sqrt{\frac{1}{|T|}\sum_{(u,i)\in T} |\hat{r}_{ui} - r_{ui}|}$

\subsection{Проверочные данные}


\subsection{Процесс тестирования}

\subsection{Результаты}





\end{document}

